\documentclass{beamer}
\mode<presentation> {
\usetheme{tcd}
\usepackage{url}
\usepackage{listings}
\usepackage{color}
\usepackage{verbatim}
\usepackage{graphicx}

\definecolor{dkgreen}{rgb}{0,0.6,0}
\definecolor{gray}{rgb}{0.5,0.5,0.5}
\definecolor{mauve}{rgb}{0.58,0,0.82}

%\lstset{frame=tb,
%  language=C,
%  aboveskip=3mm,
%  belowskip=3mm,
%  showstringspaces=false,
%  columns=flexible,
%  basicstyle={\small\ttfamily},
%  numbers=none,
%  numberstyle=\tiny\color{gray},
%  keywordstyle=\color{blue},
%  commentstyle=\color{dkgreen},
%  stringstyle=\color{mauve},
%  breaklines=true,
%  breakatwhitespace=true,
%  tabsize=3
%}
}

\title[]{Infrastructure Automation with Salt} 

\author{Sean McGrath}
\institute[TCD]
{
Resarch IT \\
Trinity College Dublin \\
\medskip
\textit{sean.mcgrath@tcd.ie} 
}
\date{10th November 2016}

\begin{document}

\begin{frame}
\titlepage
\end{frame}

%\begin{frame}
%\frametitle{Overview}

%\tableofcontents

%\end{frame}


\begin{frame}
\frametitle{About Research IT}

\begin{itemize}
\item Where I work as a systems administrator
\item http://www.tchpc.tcd.ie/
\item Ireland's premier High Performance Computing Centre with large scale Supercomputing and Visualisation facilities.
\item Assisting Researchers with computationally complex problems.
\item Previously The Trinity Centre for High Performance Computing, TCHPC
\end{itemize}

\end{frame}

\begin{frame}
\frametitle{The need to automate}

Why: Research IT has in the region of 500 linux machines, (physical and virtual).\\~\\
How: Salt *\\~\\

* previously ansible and puppet

\end{frame}

\begin{frame}
\frametitle{Why use an automation tool?}

It allows you to be:
\begin{itemize}
\item Scalable: \emph{~400 HPC nodes with identical configuration}
\item Granulated: \emph{specific configurations can be applied to a specific node(s)}
\item Automated: \emph{you write your salt statements once but call them repeatedly}
\item Repeatable: \emph{it's the same salt statement that is applied each time}
\item Efficient: \emph{once configured you don't have to intervene with specific machines}
\end{itemize}
\end{frame}

\begin{frame}[fragile] % fragile needed when quoting code or using verbatim, etc
\frametitle{Use Case 1 - re-install a 100 node cluster}

Step 1. Re-install OS on nodes with PXE \\
Step 2. Install and configure salt on nodes\\
\begin{lstlisting}[basicstyle=\tiny,]
salt-ssh --ignore-host-keys --passwd xxxxxxxx kelvin-n* state.sls installsalt
\end{lstlisting}
Step 3 - reboot node's\\
\begin{lstlisting}[basicstyle=\tiny,]
salt-ssh --ignore-host-keys --passwd xxxxxxxx kelvin-n* cmd.run 'reboot' 
\end{lstlisting}
Comment\\
\begin{itemize}
\item Salt scales very well in this instance, much better than Ansible
\item It offers great efficiencies, once developed your salt state's automate all your work for you
\item Can be easily repeated in future\\~\\
\end{itemize}
Lets look at what exactly the $installsalt$ state does.
\end{frame}


\begin{frame}[fragile]
\frametitle{Use Case 1 - $installsalt$ state}
\begin{lstlisting}[basicstyle=\tiny,]
install epel:
  pkg:
    - installed
    - pkgs:
      - epel-release
      - yum-conf-epel
salt-minion:
  pkg:
    - installed
create folders for keys:
  file:
    - name: /etc/salt/pki/minion/minion_master.pub
    - managed
    - makedirs: True

/etc/salt/pki/minion/{{ file }}:
  file:
    - managed
    - source: salt://installsalt/{{ file }}.{{ cluster_name }}
    - watch_in:
      - service: restart salt minion

restart salt minion:
  service:
    - name: salt-minion
    - running
/etc/rc.local:
  file:
    - append
    - text: "sleep 10; salt-call state.highstate pillar='{\"reboot\": \"yes\"}' #wait 10 secs for GPFS"
\end{lstlisting}
\end{frame}

\begin{frame}[fragile]
\frametitle{Use Case 2 - Bootstrap installation}
Scenario: new VM and want to put your common config on it.\\
Assumptions:\\
\begin{itemize}
\item OS and salt installed on VM
\item minions key signed by master
\end{itemize}
From the salt master:\\
\begin{lstlisting}[basicstyle=\tiny,]
salt newserver.fqdn state.sls bootstrap
\end{lstlisting}
Which does:\\
\begin{lstlisting}[basicstyle=\tiny,]
include:
  - general.init # incase a highstate isn't called ensure the states that should apply to all hosts are applied
  - tchpc-general.epel
  - tchpc-general.repositories.local
  - tchpc-general.nrpe
  - tchpc-general.shorewall
  - tchpc-general.postfix
  - tchpc-general.snmp
  - tchpc-general.services
  - tchpc-general.rsyslog
  - tchpc-general.check_updates
  - tchpc-general.bacula
\end{lstlisting}
Why: your standard setup is repeatedly and automatically applied.
\end{frame}

\begin{frame}
\frametitle{Use Case 3 - software upgrade testing}
Scenario:\\
\begin{itemize}
\item kernel needs to be updated.
\item Software, (GPFS - parallel file system) depends on kernel version.
\item Both need to be updated simultaneously.
\item Want to test on a subset of nodes first.
\end{itemize}
Tell the minion what version to use - pillar\\
\begin{quote}
Pillar is an interface for Salt designed to offer global values that can be distributed to minions.
\footnote{\url{https://docs.saltstack.com/en/carbon/topics/pillar/index.html}}
\end{quote}
Identify the minion to apply the pillar variable to - grains\\
\begin{quote}
interface to derive information about the underlying system. This is called the grains interface
\footnote{\url{https://docs.saltstack.com/en/latest/topics/grains/}}
\end{quote}
\end{frame}

\begin{frame}[fragile]
\frametitle{Use Case 3 - install specific versions on specific node}

Set the pillar to the updated versions for your test node:\\

\begin{lstlisting}[basicstyle=\tiny,]
# testing upgraded versions on specific nodes:

gpfs_version: 3.5.0-32

gpfs_version: 3.5.0-29


kernel_version: 2.6.32-642.3.1.el6.x86_64

kernel_version: 2.6.32-573.12.1.el6.x86_64

\end{lstlisting}

Install the relevant kernel version, (pillar variable) for the node, (identified by grain).

\begin{lstlisting}[basicstyle=\tiny,]

install kernel packages (cmd):
  cmd:
    - run
    - name: yum -y install kernel-headers-{{ salt['pillar.get']('kernel_version') }} kernel-{{ salt['pillar.get']('kernel_version') }} kernel-devel-{{ salt['pillar.get']('kernel_version') }}
\end{lstlisting}

This provides excellent granularity without having to provision a test environment.

\end{frame} 

\begin{frame}
\frametitle{Use Case 4 - GPU card installation}
Process:\\
\begin{enumerate}% itemized / ordered list
\item Remove unsupported kernel modules and reboot to load correct kernel modules.
\item Generate a new ramdisk without the supported kernel modules.
\item Boot from that new ramdisk.
\item Install the GPU drivers and reboot.\\~\\ % leave a line blank after this for prettier spacing
\end{enumerate}

Limitation: Only want to run this state on a node with GPU hardware.\\~\\

Gotcha: Possible infinite loop of reboots unless the minion knows each step has completed successfully.\\~\\

Solution: Set a grain after each step.
\end{frame}

\begin{frame}[fragile]
\frametitle{Use Case 4 - Continued}


Ensure this state only runs on a node with the gpu installed in it:\\
\begin{lstlisting}[basicstyle=\tiny,]

\end{lstlisting}

Unload kernel modules 
\begin{lstlisting}[basicstyle=\tiny,]
# the nouveu modules need to be removed from the kernel
/etc/modprobe.d/blacklist-nouveau.conf:
  file.managed:
    - source: salt://clusters/nodes/gpu-boole/blacklist-nouveau.conf
    - mode: 644
    - user: root
    - group: root
\end{lstlisting}
Reboot, requires salt being called with a reboot pillar set
\begin{lstlisting}[basicstyle=\tiny,]

always-fails gpu:
  test.fail_without_changes:
    - name: MESSAGE - the minion should reboot 
    - failhard: True
 # end if pillar['reboot'] != 'yes' %}
\end{lstlisting}
\end{frame}

\begin{frame}[fragile]
\frametitle{Use Case 4 - Continued}

Generate the new ramdisk from those modules\\

\begin{lstlisting}[basicstyle=\tiny,]

# ramdisk needs to be re-generated without the nouveu modules and node booted from it
create ramdisk without the nouveau modules:
  cmd.run:
    - name: dracut --force
\end{lstlisting}

Set a grain value on the minion to say that ramdisk has been re-generated.\\

\begin{lstlisting}[basicstyle=\tiny,]
regenerate_ramdisk:
  module.run:
    - name: grains.setval
    - key: regenerate_ramdisk
    - val: 'regenerated'
\end{lstlisting}

Boot from the new ramdisk\\

\begin{lstlisting}[basicstyle=\tiny,]
system.reboot ramdisk:
  module:
    - name: system.reboot
    - run
    - require:
      - module: regenerate_ramdisk
stops after ramdisk reboot:
  test.fail_without_changes:
    - name: MESSAGE - system rebooting
    - failhard: True
    - require:
      - module: system.reboot ramdisk

\end{lstlisting}
\end{frame}

\begin{frame}[fragile]
\frametitle{Use Case 4 - Continued}

Install GPU drivers only if they haven't all ready been installed \\

\begin{lstlisting}[basicstyle=\tiny,]

Install Nvidia drivers:
  cmd.run:
    - name: /home/support/root/gpu/cuda_7.5.18_linux.run --silent
\end{lstlisting}

Set the grain to say they've been installed to prevent a reboot loop and reboot:\\

\begin{lstlisting}[basicstyle=\tiny,]
nvidia_drivers:
  module.run:
    - name: grains.setval
    - key: nvidia_drivers
    - val: 'installed'
system.reboot nvidia_drivers:
  module:
    - name: system.reboot
    - run
    - require:
      - module: nvidia_drivers
stops after nvidia drivers reboot:
  test.fail_without_changes: # this is really supported only from Salt 2014.7
    - name: MESSAGE - system rebooting
    - failhard: True
    - require:
      - module: system.reboot nvidia_drivers

\end{lstlisting}
\end{frame}

\begin{frame}
\frametitle{Use Case 4 - benefits}

Salt provides a simple and easy way to automatically provision a complex installation.\\~\\

It is easily repeated if a minion has to be re-installed or new machines added.\\~\\

You have a centralise documentation store of what exactly needs to be done to set your installation up.

\end{frame}

\begin{frame}
\frametitle{Salt take aways}
\begin{itemize}
\item Salt does the work of configuring your machines for you
\item Salt provides system documentation
\item Salt provides a knowledge base of "How to do X"
\end{itemize}
\end{frame}

\begin{frame}[fragile]
\frametitle{Thank You!}
Slide source available at: \url{https://github.com/smcgrat/Presentations/blob/master/Infrastructure_Automation_with_SaltStack.tex} \\~\\
Questions?
\end{frame}
\end{document}
